\chapter{Conclusioni}
\label{cap:conclusioni}

In questa tesi è stata condotta una ricerca estensiva volta a studiare l'abilità dei sistemi di machine learning nel distinguere le attività di lavaggio e sanificazione delle mani di gran parte della popolazione utilizzando strumenti di misurazione non invasivi come i comuni smartwatch reperibili commercialmente. In particolare ci siamo concentrati sul riconoscimento dei gesti di lavaggio partendo da dati non strutturati, ovvero prodotti da utenti non istruiti sulla corretta maniera in cui lavare o sanificare le mani i quali sono stati lasciati liberi di usare il loro metodo di lavaggio abituale. Durante la ricerca siamo giunti alla creazione di un dataset personalizzato utilizzando una IMU posizionata sul polso dominante per raccogliere svariate ore di dati durante attività di vita reale dai diversi soggetti.

Il dataset costruito ad hoc, assieme ad un dataset DLA disponibile pubblicamente sono stati utilizzati per addestrare i diversi modelli di machine learning e di deep learning durante una serie di esperimenti condotti sia su ambiente PC che direttamente sullo smartwatch scelto: l'HEXIWEAR. Questi esperimenti hanno permesso di valutare le performance delle diverse tipologie di modelli su un ambiente dalle risorse pressoché illimitate, quello PC, ma anche su uno smartwatch le cui risorse sono altamente limitate, sia in termini di memoria che di potenza di calcolo e di durata della batteria. Dalla ricerca è emerso che in ambente PC e con il dataset costruito ad hoc i modelli hanno prodotto metriche di accuratezza con valori superiori al 93\%, mentre sul dispositivo HEXIWEAR, a causa delle minori risorse, le performance sono scese attorno all'87\%; ciò dimostra che le operazioni di riconoscimento e monitoraggio di queste attività tramite dati inerziali raccolti dal polso dell'utente mediante un comune smartwatch sono totalmente fattibili e producono ottimi risultati, in linea con gli studi sul riconoscimento delle altre tipologie di attività umane, come la corsa e la camminata.

Nell'ultima parte di questa tesi abbiamo presentato lo sviluppo di un'applicazione nativa per lo smartwatch HEXIWEAR con lo scopo di monitorare costantemente l'igiene delle mani dell'utente che lo indossa; per fare questo i sensori di accelerometro e giroscopio sono campionati costantemente ed il modello di deep learning LSTM è usato per classificare le corrette attività aggiornando l'interfaccia grafica.

Uno dei potenziali sviluppi futuri di questo lavoro consiste nell'approfondire la ricerca svolta in tutti gli ambiti, in particolare ampliando il dataset costruito ad hoc raccogliendo ancora più ore di dati da un maggior numero di soggetti di genere ed età diversi; in questo modo si va a creare un dataset pubblico di ottima qualità specifico per le misurazioni delle attività non strutturate di lavaggio e sanificazione delle mani, che, dalle nostre ricerche, non sembra ancora esistere.

La ricerca può essere ampliata anche nell'ambito dei sistemi di machine learning svolgendo molti più test, ad esempio eseguendoli direttamente sullo smartwatch utilizzando i dati raccolti dai suoi sensori inerziali in tempo reale. 

Infine un’ultima area di miglioramento è la qualità dei modelli portati sullo smartwatch; a causa delle grandi limitazioni in termini di memoria siamo stati costretti a ridurre severamente la dimensione dei modelli eseguiti sull'HEXIWEAR. Uno sviluppo futuro prevede sicuramente l’analisi dei dispositivi più performanti in grado di eseguire gli stessi modelli studiati nell’ambiente PC.